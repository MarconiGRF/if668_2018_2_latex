\documentclass[10pt]{article}

\usepackage[utf8]{inputenc}
\usepackage[brazil]{babel}
\usepackage[T1]{fontenc}

\title{IF702 - Redes Neurais}
\author{Marconi Gomes}
\date{Outubro 2018}

\usepackage{natbib}
\usepackage{graphicx}
\usepackage{placeins}
\usepackage[table,xcdraw]{xcolor}

%% Pagina e Margens.
\usepackage[a4paper,top=3cm,bottom=2cm,left=3cm,right=3cm,marginparwidth=1.75cm]{geometry}


\begin{document}

\maketitle


\section{Introdução}
A disciplina de Redes Neurais utiliza de fundamnetos matemáticos e técnicas como o estudo de Redes feedfoward e Redes recorrentes para a criação de modelos de inteligência e desenvolvimento de sistemas baseados também em inteligência computacional. Esses modelos então são usados para a resolução de problemas reais, além de criar formas de habilitar o computador a realizar tarefas onde o ser humano tem um melhor desempenho, podendo ser aplicado em um número muito abrangente de áreas de estudo. A disciplina se insere como parte fundamental da área de Inteligência Artificial e Aprendizagem de Máquina. \cite{cin}

\begin{figure}[h!]
\centering
\includegraphics[scale=2.8]{Rede-Neural-UFCG.png}
\caption{Exemplo de Rede Neural \cite{ufcg}}
\label{fig:Rede-Neural-UFCG}
\end{figure}

\section{Relevância}
  Uma vez que a disciplina possibilita o aprendizado de máquina, a disciplina se torna essencial para a resolução de problemas reais contextuados em sociedade, consequentemente se tornando relevante para o curso de Ciência da Computação em geral pela sua abrangência, e especialmente para o graduando que pensa em especializar-se na área de Inteligência Artificial.\cite{gcv}

\newpage
\section{Relação com outras disciplinas}
\begin{table}[h]
\footnotesize
\begin{tabular}{ll}
\hline
\rowcolor[HTML]{9B9B9B} 
\multicolumn{2}{c}{\cellcolor[HTML]{9B9B9B}\textbf{Interdisciplinaridade}}          
\\
\hline
\rowcolor[HTML]{C0C0C0} 
\multicolumn{1}{|c|}{\cellcolor[HTML]{C0C0C0}{\color[HTML]{333333} Código/Nome da Disciplina}}                    & \multicolumn{1}{c|}{\cellcolor[HTML]{C0C0C0}{\color[HTML]{333333} Relação com Redes Neurais}}                                                                                                                                                                                                                                                                                                                                     \\ \hline
\multicolumn{1}{|l|}{IF670 - Matemática Discreta}                                                                 & \multicolumn{1}{l|}{\begin{tabular}[c]{@{}l@{}}A disciplina de Redes Neurais tem correlação com Matemática discreta pois\\ usa noções de conjuntos para analisar algoritmos de aprendizagem, onde é feita\\ a análise de elementos que pertencem a certos padrões de comportamento. \cite{md}\end{tabular}}                                                                                                   \\ \hline
\multicolumn{1}{|l|}{\begin{tabular}[c]{@{}l@{}}MA531 - Álgebra Vetorial e Linear\\ para Computação\end{tabular}} & \multicolumn{1}{l|}{\begin{tabular}[c]{@{}l@{}}A disciplina de Redes Neurais usa noções de combinação e regressão linear, bem\\ como matrizes para um método de treinamento de aprendizado (mais conhecido\\ como pseudo-inverse method), que são amplamente estudadas no primeiro\\ período de Ciência da Computação na disciplina de Álgebra Linear e Vetorial\\ para Computação. \cite{avlc}\end{tabular}} \\ \hline

\multicolumn{1}{|l|}{\begin{tabular}[c]{@{}l@{}}IF673 - Lógica para Computação \end{tabular}} &
\multicolumn{1}{l|}{\begin{tabular}[c]{@{}l@{}}A disciplina de Redes Neurais usa conceitos básicos de lógica computacional\\ para a composição da estrutura de controle dos "neurônios computacionais",\\ bem como seu sistema de controle de decisão; tais conceitos básicos de lógica\\ são estudados no segundo período, na disciplina de lógica para computação. \cite{gcv2}\end{tabular}} \\ \hline
\end{tabular}
\end{table}

\FloatBarrier
\bibliographystyle{plain}
\bibliography{references}
\end{document}
